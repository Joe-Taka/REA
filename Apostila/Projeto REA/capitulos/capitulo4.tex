\section{Transformada de Fourier}

\subsection{O que é a Transformada de Fourier?}

A transformada de Fourier é uma transformação matemática que "quebra" uma forma de onda (função ou sinal) em uma soma de funções periódicas. A análise de Fourier converte um sinal do seu domínio original para uma representação no domínio da frequência.

Jean-Baptiste Joseph Fourier (Auxerre, 21 de março de 1768 — Paris, 16 de maio de 1830) foi o matemático e físico francês que descobriu que todo sinal pode ser descrito como uma superposição de senóides complexas.

\subsection{Sinais}
Os sinais traduzem a evolução de uma grandeza ao longo do tempo ou espaço e podem ser classificados segundo os critérios:

\begin{itemize}
\item Continuidade: sinais contínuos ou discretos
\item Periodicidade: sinais periódicos ou aperiódicos 
\end{itemize}

\subsubsection{Sinais contínuos}

Um sinal diz-se contínuo se o seu domínio for IR ou um intervalo contínuo de IR. Assim:

\[ x: IR \rightarrow IR \]

\[ x: [a,b] \rightarrow IR \]

- Sinais contínuos são sinais que não possuem espaços distinguíveis entre os seus valores.
- Advém de grandezas que podem ser medidas.

\subsubsection{Sinais discretos}
\begin{itemize}
\item Sinais discretos são sinais que possuem espaços entre os seus valores.
\item Advém de grandezas que são contadas.
\item Só os sinais discretos podem ser armazenados e processados em computadores digitais.
\item Pode-se converter um sinal contínuo num sinal discreto através da coleção de amostras do sinal contínuo.
\end{itemize}

\subsubsection{Sinais periódicos}
Um sinal discreto x(n) é periódico com período $N \ \in \ IN$ se:

\[ \large x(n + N) = x(n), \forall \ n \ \in Z \]

\subsection {Tipos de Abordagens}

De acordo com o tipo de sinal que estamos lidando, existem 4 diferentes tipos de abordagem:
\begin{itemize}
\item Sinais contínuos e aperiódicos: Transformada de Fourier
\item Sinais contínuos e periódicos: Série de Fourier
\item Sinais discretos e aperiódicos: Transformada de Fourier de Tempo Discreto (TFTD)
\item Sinais discretos e periódicos: Transformada Discreta de Fourier (TDF)
\end{itemize}

