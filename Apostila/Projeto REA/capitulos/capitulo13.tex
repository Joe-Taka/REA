\section{Integração Numérica - Quantidade de calor}

Nas engenharias e nas ciências, frequentemente se quer determinar o quanto de calor precisa ser fornecido, a uma determinada substância, para que esta saia de uma temperatura inicial para uma temperatura final. Para isso é necessário determinar a capacidade calorifica do material em questão. A fórmula abaixo mostra o cálculo da quantidade de calor utilizando a capacidade calorífica:

$$\Large Q = mc \Delta T$$

\begin{itemize}
\item Onde $Q$ é a quantidade de calor, $m$ a massa em questão,  $c$ a capacidade calorífica e delta $\Delta T$ a variação de temperatura desejada.
\item $c$ diz respeito a quantidade de energia necessária que deve ser fornecida (ou retirada do) ao material para que uma unidade de massa desse material varia sua temperatura em uma unidade de temperatura.
\end{itemize}

A capacidade calorífica é definida de duas formas, capacidade calorífica a volume constante $C_{V}$ e a capacidade calorifica a pressão constante, $C_{P}$.

Entretanto, um problema prático surge. A capacidade calorífica é constante (não depende da temperatura) para pequenos intervalos de temperatura, mas para intervalos maiores $c$ já não é mais constante e passa a variar com a temperatura sob a forma da seguinte equação empírica:

$$\Large \frac{C_{P}}{R} = A + BT + CT^2+DT^{-2}$$

Os parâmetros $A$ , $B$, $C$ e $D$ são independentes da temperatura, mas acabem por sofrer influência do valor da pressão constante. Normalmente, os dois últimos parâmetros são iguais a zero para uma grande gama de substâncias.

Visto que a fração do lado esquerdo da equação acima é adimensional, as unidades de  $C_{P}$ são as escolhidas para a constante universal dos gases $R$.

Vamos determinar a a quantidade de calor necessária para elevar a temperatura da amônia de 10°C à 100°C utilizando integração numérica no Python.

Existe uma biblioteca da linguagem Python chamada Scipy voltada para ciências e engenharias que possui o pacote de integração \textbf{scipy.integrate}. Nesse pacote utilizaremos a função \textbf{quad} que recebe como parâmetros a função a ser integrada, e o intervalo de integração delimitados pelo valor inicial $a$ e o valor final $b$. Essa função retorna ainda, o valor numérico da integração e o uma estimativa do erro numérico associado ao método de integração utilizado.

Os valores dos parâmetros $A$, $B$, $C$ e $D$ para Amônia são retirados do livro Introdução à Termodinâmica da Engenhara Química (SMITH, J.M. et al), mais especificamente na tabela C.3 do apêndice C.

$$\Large A=22,626; B=-100,75.10^{-3}; C = 192,71.10^{-6}$$

Sendo assim, temos que buscamos integrar a seguinte expressão:

$$\Large \Delta H = \int_{283K}^{373K} \frac{C_{p}}{R}dt$$

$$\Large \Delta H = \int_{283K}^{373K} (22,626 - 100,75.10^{-3}T + 192,71.10^{-6}T^2)dT$$

\begin{minted}{Scilab}
from scipy.integrate import quad
#importa a função quad para integrar numericamente uma função

#definindo a função em questão para utilizar 
#na função quad
def funcao(T):
#função de Cp para a amônia 
return 22.626 - 0.10075*T + 0.00019271*(T**2) 

# 10 °C e 100°C na escala Kelvin respectivamente
c, erro = quad(funcao, 283, 373) 

print("o resultado é: {:f} (+-{:g})"
# é necessário multplicar o resultado 
# pelo valor da Constante universal dos gases R = 8.413
.format(c*8.314, erro)) 
\end{minted}